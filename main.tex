%% 
%% Copyright 2007-2020 Elsevier Ltd
%% 
%% This file is part of the 'Elsarticle Bundle'.
%% ---------------------------------------------
%% 
%% It may be distributed under the conditions of the LaTeX Project Public
%% License, either version 1.2 of this license or (at your option) any
%% later version.  The latest version of this license is in
%%    http://www.latex-project.org/lppl.txt
%% and version 1.2 or later is part of all distributions of LaTeX
%% version 1999/12/01 or later.
%% 
%% The list of all files belonging to the 'Elsarticle Bundle' is
%% given in the file `manifest.txt'.
%% 
%% Template article for Elsevier's document class `elsarticle'
%% with harvard style bibliographic references

%\documentclass[preprint,12pt,authoryear]{elsarticle}

%% Use the option review to obtain double line spacing
%% \documentclass[authoryear,preprint,review,12pt]{elsarticle}

%% Use the options 1p,twocolumn; 3p; 3p,twocolumn; 5p; or 5p,twocolumn
%% for a journal layout:
%% \documentclass[final,1p,times,authoryear]{elsarticle}
%% \documentclass[final,1p,times,twocolumn,authoryear]{elsarticle}
%% \documentclass[final,3p,times,authoryear]{elsarticle}
%% \documentclass[final,3p,times,twocolumn,authoryear]{elsarticle}
%% \documentclass[final,5p,times,authoryear]{elsarticle}
 \documentclass[final,5p,times,twocolumn,authoryear]{elsarticle}

%% For including figures, graphicx.sty has been loaded in
%% elsarticle.cls. If you prefer to use the old commands
%% please give \usepackage{epsfig}

%% The amssymb package provides various useful mathematical symbols
%-- coding: UTF-8 --
\usepackage[UTF8]{ctex}
\usepackage{amssymb}
\usepackage{amsmath}
\usepackage{lipsum}
%% The amsthm package provides extended theorem environments
%% \usepackage{amsthm}

%% The lineno packages adds line numbers. Start line numbering with
%% \begin{linenumbers}, end it with \end{linenumbers}. Or switch it on
%% for the whole article with \linenumbers.
%% \usepackage{lineno}

%% You might want to define your own abbreviated commands for common used terms, e.g.:
\newcommand{\kms}{km\,s$^{-1}$}
\newcommand{\msun}{$M_\odot}

\journal{Astronomy $\&$ Computing}


\begin{document}

\begin{frontmatter}

%% Title, authors and addresses

%% use the tnoteref command within \title for footnotes;
%% use the tnotetext command for theassociated footnote;
%% use the fnref command within \author or \affiliation for footnotes;
%% use the fntext command for theassociated footnote;
%% use the corref command within \author for corresponding author footnotes;
%% use the cortext command for theassociated footnote;
%% use the ead command for the email address,
%% and the form \ead[url] for the home page:
%% \title{Title\tnoteref{label1}}
%% \tnotetext[label1]{}
%% \author{Name\corref{cor1}\fnref{label2}}
%% \ead{email address}
%% \ead[url]{home page}
%% \fntext[label2]{}
%% \cortext[cor1]{}
%% \affiliation{organization={},
%%            addressline={}, 
%%            city={},
%%            postcode={}, 
%%            state={},
%%            country={}}
%% \fntext[label3]{}

\title{Title of paper}

%% use optional labels to link authors explicitly to addresses:
%% \author[label1,label2]{}
%% \affiliation[label1]{organization={},
%%             addressline={},
%%             city={},
%%             postcode={},
%%             state={},
%%             country={}}
%%
%% \affiliation[label2]{organization={},
%%             addressline={},
%%             city={},
%%             postcode={},
%%             state={},
%%             country={}}

\author[first]{Author name}
\affiliation[first]{organization={University of the Moon},%Department and Organization
            addressline={}, 
            city={Earth},
            postcode={}, 
            state={},
            country={}}

\begin{abstract}
%% Text of abstract
Example abstract for the astronomy and computing journal. Here you provide a brief summary of the research and the results.
\end{abstract}

%%Graphical abstract
%\begin{graphicalabstract}
%\includegraphics{grabs}
%\end{graphicalabstract}

%%Research highlights
%\begin{highlights}
%\item Research highlight 1
%\item Research highlight 2
%\end{highlights}

\begin{keyword}
%% keywords here, in the form: keyword \sep keyword, up to a maximum of 6 keywords
keyword 1 \sep keyword 2 \sep keyword 3 \sep keyword 4

%% PACS codes here, in the form: \PACS code \sep code

%% MSC codes here, in the form: \MSC code \sep code
%% or \MSC[2008] code \sep code (2000 is the default)

\end{keyword}


\end{frontmatter}

%\tableofcontents

%% \linenumbers

%% main text

\section{AMFR-CAV跟驰模型的建立}
\label{introduction}

1995年,Bando等提出了(Optimal Velocity ,OV)模型,其微分方程为:
\begin{equation}\label{eqn:1}
    \frac{\mathrm{d}}{\mathrm{d}t}v_{_{n} } (t)=a[V(\Delta  x_{n} )-v_{n}(t)]
\end{equation}
\par
(1)式中$v_{_{n} } (t)$为跟驰模型中的第n辆车的速度;$\Delta x_n(t)=x_{n+1}(t)-x_{n}(t)$为第n辆车(跟驰车)与第n+1辆车(前车)的车间距;$V(\Delta  x_{n} )$为期望速度函数。
\par
1998年,Helbing和Tilch根据数据提出$V(\Delta  x_{n} )$函数:
\par
\begin{equation}
    V(\Delta  x_{n} )=V_{1}+V_{2}\tan{h}[C_{1}(\Delta x_{n}-C_{2}]
\end{equation}
该式中,$V_{1}=6.75m/s, V_{2}=7.91m/s, C_{1}=0.13m^{-1}, l_{c}=5m, C_{2}=1.57$
\par
由(2)式中可知,当车间距$\Delta x_{n}(t)$非常小时,期望速度$V(\Delta  x_{n} )$也就非常的小。当$V(\Delta  x_{n} )<v_{n}(t)$时,车辆将减速。可在实际交通中,当前车的速度比跟驰车辆速度快的很多时,尽管车间距$V(\Delta  x_{n} )$很小,跟驰车辆人会加速,OV模型中缺少前车速度变量。
\par
Jiang等提出了全速度差模型(Full Velocity Difference, FVD),如下:
\begin{equation}
    \frac{\mathrm{d}}{\mathrm{d}t}v_{_{n} } (t)=a[V(\Delta  x_{n} )-v_{n}(t)]+\lambda\Delta v_{n}
\end{equation}
该式中,当前车速度大于跟驰车速,$\Delta V_{n}(t)$为正,否则为负。FVD模型通过引入变量$\Delta V_{n}$,模拟出符合实际堵塞情况下的车辆启动延迟时间和博速度。
\par
过去的研究基于人们对驾驶环境的感知有限,对前后车距离、速度、加速度的变化敏感性因人而异,跟驰模型通常只考虑前车数量最多为2辆,后车数量为1辆。但随着CAV环境的出现,跟驰车辆通过传感器可以探测到紧邻前后车辆的位置和运动状态,而CAV之间可以通过通信设备交流,使得跟驰车辆可以获得前后多车的位置和运动状态,实现网联式驾驶辅助功能。目前,基于CAV环境的研究仅限于通过引入多前后车辆信息,探究模型对交通流稳定性的影响。本研究考虑到实际情况下CAV跟驰车辆能够获取前后车数量信息的限制,探究不同的前后车数量对交通流稳定性的影响。现有文献中通常认为,引入前车数量越多,曲线脉动幅度越小,交通流稳定性越好,但缺乏CAV环境中对此结论的验证。
\par

为了探究不同前后车数量对交通流稳定性的影响,本研究结合了FVD模型、孙棣华等人提出的考虑后视效应和速度差信息的跟驰模型(BLVD)以及宗芳等人提出的考虑前后多车信息的混行跟驰模型(Multi⁃Front and Rear Headway Velocity and Acceleration Difference, MFRHVAD)。这些模型都考虑了更多的前后车信息,能够更加准确地描述车辆间的跟驰行为,从而有助于深入探究前后车数量对交通流稳定性的影响。基于CAV提出考虑多车头间距$\Delta X_{n}(t)$、速度差$\Delta V_{n}(t)$和加速度差$\Delta a_{n}(t)$的前后不对称网联多车更驰模型(AMFR-CAV):
\begin{equation}
    \begin{split}
           \frac{\mathrm{d}}{\mathrm{d}t}v_{_{n} } (t)
           &=a[pV_{F}(\sum_{j=1}^P\alpha_{Fj}\Delta x_{n+j-1}(t))\\
           &+(1-P)V_{B}(\sum_{m=1}^Q\alpha_{Bm}\Delta x_{n-m}(t))-v_{n}(t)]\\
           &+b\sum_{j=1}^P\beta_{j}\Delta v_{n+j-1}(t)+c\sum_{j=1}^{P}\Delta a_{n+j-1}(t-1)
    \end{split}
\end{equation}
\par
该式中,a,b,c分别为网联车的期望函数敏感系数、速度差敏感系数与加速度差敏感系数, 这里 a=0.05,b=0.2,c=0.1。由于跟驰车辆受前方车辆的影响程度比受到后方车辆的影响程度小,p为前车期望速度函数权重取值0.9,则后车期望速度函数权重取值0.1。$V_{F}$为前车的期望速度;$V_{B}$为候车的期望速度。j为相对跟驰车辆的第前j辆车;m为相对跟驰车辆的第后辆车;$\Delta x_{n+j-1}$为前j辆车与前j+1辆车道车头间距;$\Delta v_{n-m}$为后m+1辆车的车头间距;$\Delta v_{n}$为第n+1辆车(前车)与第n辆车后车的速度差;$\Delta a_{t}$为第n+1辆车(前车)与第n辆车后车的加速度差;$\alpha_{Fj}$为对;$\Delta x_{n+j-1}$的权重;$\alpha_{Bm}$为对$\Delta v_{n-m}$的权重;$\beta_{j}$为$\Delta v_{n}$的权重。因为$\sum_{j=1}^P\alpha _{Fj}=1\sum_{m=1}^q\alpha _{Bm}=1,\sum_{j=1}^p\beta_{j}=1,\sum_{j=1}^p\lambda_{j}=1$o所以对$\alpha_{Fj}$,$\alpha_{Bm}$,$\beta_{j}$,$\lambda_{j}$的方式为:
\par
	\begin{eqnarray}
		\alpha_{Fj}=\beta_{j}=\lambda_{j}=\left\{
		\begin{aligned}
			\frac{P-1}{P^{i}} &  &{j\neq P} \\
			\frac{1}{p^{i-1}} &  &{j=P}     \\
		\end{aligned}
		\right.
	\end{eqnarray}
\par
该式中P为模型中考虑的前车数量
\par
	\begin{eqnarray}
		\alpha_{Bm}=\left\{
		\begin{aligned}
			\frac{Q-1}{Q^{m}} &  &{m\neq Q} \\
			\frac{1}{Q^{m-1}} &  &{m=Q}     \\
		\end{aligned}
		\right.
	\end{eqnarray}
 \par
 该式中Q为模型中考虑的候车数量,且P+Q=10。
 \par
 AMFR-CAV模型中前车、后车期望速度函数如式(7)、式(8)
 \par
 \begin{equation}
     V_{F}(\Delta x_{n})=V_{1}+\tan h[C_{1}(\Delta x_{n}-l_{c}-C_{2}]
 \end{equation}
  \par
 \begin{equation}
     V_{B}(\Delta x_{n})=-V_{1}+\tan h[C_{1}(\Delta x_{n}-l_{c}-C_{2}]
 \end{equation}
 
\section{Title 2}
%%\label{}
\lipsum[1]

\subsection{Subsection title}

\begin{figure}
	\centering 
	\includegraphics[width=0.4\textwidth, angle=-90]{ASCOM_journal_cover.pdf}	
	\caption{Astronomy \& Computing journal cover} 
	\label{fig_mom0}%
\end{figure}

A random equation, the Toomre stability criterion:

\begin{equation}
    Q = \frac{\sigma_v \times \kappa}{\pi \times G \times \Sigma}
\end{equation}

\section{Title 3}
%%\label{}
\lipsum[2]

\subsection{Subsection title}
\lipsum[3]

\begin{table}
\begin{tabular}{l c c c} 
 \hline
 Source & RA (J2000) & DEC (J2000) & $V_{\rm sys}$ \\ 
        & [h,m,s]    & [o,','']    & \kms          \\
 \hline
 NGC\,253 & 	00:47:33.120 & -25:17:17.59 & $235 \pm 1$ \\ 
 M\,82 & 09:55:52.725, & +69:40:45.78 & $269 \pm 2$ 	 \\ 
 \hline
\end{tabular}
\caption{Random table with galaxies coordinates and velocities, Number the tables consecutively in
accordance with their appearance in the text and place any table notes below the table body. Please avoid using vertical rules and shading in table cells.
}
\label{Table1}
\end{table}


\section{Discussion}
%%\label{}
\lipsum[4]

\section{Summary and conclusions}
%%\label{}
\lipsum[1-4]


\section*{Acknowledgements}
Thanks to ...

%% The Appendices part is started with the command \appendix;
%% appendix sections are then done as normal sections
\appendix

\section{Appendix title 1}
%% \label{}

\section{Appendix title 2}
%% \label{}

%% If you have bibdatabase file and want bibtex to generate the
%% bibitems, please use
%%
\bibliographystyle{elsarticle-harv} 
\bibliography{example}

%% else use the following coding to input the bibitems directly in the
%% TeX file.

%%\begin{thebibliography}{00}

%% \bibitem[Author(year)]{label}
%% For example:

%% \bibitem[Aladro et al.(2015)]{Aladro15} Aladro, R., Martín, S., Riquelme, D., et al. 2015, \aas, 579, A101


%%\end{thebibliography}

\end{document}

\endinput
%%
%% End of file `elsarticle-template-harv.tex'.
